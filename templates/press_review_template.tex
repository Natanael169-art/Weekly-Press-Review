import csv
import feedparser
from datetime import datetime, timedelta
from jinja2 import Environment, FileSystemLoader, select_autoescape
import os

# Configuration
csv_file = "client_rss_feeds_cleaned.csv"
tex_output = "press_review.tex"
template_dir = "templates"
template_file = "press_review_template.tex"

# Create template directory if it doesn't exist
os.makedirs(template_dir, exist_ok=True)

# Save LaTeX template to file
latex_template = r"""
\documentclass[11pt]{article}
\usepackage[utf8]{inputenc}
\usepackage[a4paper,margin=2.5cm]{geometry}
\usepackage{lmodern}
\usepackage{hyperref}
\usepackage{xcolor}
\usepackage{titlesec}
\usepackage{fancyhdr}
\usepackage{tocloft}
\usepackage{enumitem}
\usepackage{datetime}

\pagestyle{fancy}
\fancyhf{}
\rhead{Weekly Press Review}
\lhead{\today}
\cfoot{\thepage}

\titleformat{\section}{\Large\bfseries\sffamily\color{blue}}{}{0em}{}
\titleformat{\subsection}{\bfseries\color{black}}{}{0em}{}

\setlength{\parskip}{0.5em}
\setlength{\parindent}{0pt}

\begin{document}
\tableofcontents
\newpage


\section*{{{{ company }}}}

\subsection*{{{{ article.title }}}}
\textbf{Published:} {{{{ article.published }}}} \\
\textbf{Summary:} {{{{ article.summary }}}} \\
\textbf{Link:} \href{{{{ article.link }}}}{{{{ article.link }}}}

\vspace{1em}



\end{document}
"""

with open(os.path.join(template_dir, template_file), "w", encoding="utf-8") as f:
    f.write(latex_template)

# Prepare Jinja2 environment
env = Environment(
    loader=FileSystemLoader(template_dir),
    autoescape=select_autoescape()
)
template = env.get_template(template_file)

# Date limit: 7 days ago
now = datetime.utcnow()
seven_days_ago = now - timedelta(days=7)

# Parse RSS feeds and collect articles
press_data = {}

with open(csv_file, newline='', encoding='utf-8') as f:
    reader = csv.DictReader(f)
    for row in reader:
        company = row.get("Company", "Unnamed Company")
        rss_url = row.get("RSS Feed URL", "").strip()

        if not rss_url:
            continue

        feed = feedparser.parse(rss_url)

        if feed.bozo:
            continue

        recent_entries = []
        for entry in feed.entries:
            pub_date = entry.get("published_parsed") or entry.get("updated_parsed")
            if pub_date:
                pub_datetime = datetime(*pub_date[:6])
                if pub_datetime >= seven_days_ago:
                    recent_entries.append({
                        "title": entry.get("title", "No title"),
                        "summary": entry.get("summary", "").replace("Read more", "").strip(),
                        "published": entry.get("published", "") or entry.get("updated", ""),
                        "link": entry.get("link", "")
                    })

        if recent_entries:
            press_data[company] = recent_entries[:5]  # Limit to 5 articles per company

# Render LaTeX file
rendered_tex = template.render(data=press_data)

with open(tex_output, "w", encoding="utf-8") as f:
    f.write(rendered_tex)

print(f"✅ LaTeX file generated: {tex_output}")

